% !TeX program = lualatex

\documentclass[12pt, fleqn]{article}
\setlength{\columnsep}{-9cm}

%packages
\usepackage{enumitem}
\usepackage{multicol}
\usepackage{tikz}
\usepackage[dutch]{babel}

%fonts
\usepackage[sfdefault,lining]{FiraSans} %% option 'sfdefault' activates Fira Sans as the default text font
\renewcommand*\oldstylenums[1]{{\firaoldstyle #1}}

\usepackage{amsmath}
\usepackage[mathrm=sym]{unicode-math}
\setmathfont{Fira Math}

%preamble
\date{\today}
\author{Ronan Berntsen}
\title{Wiskunde notities}
\begin{document}
\maketitle

\newcommand{\answer}[1]{\begin{multicols}{2}\textbf{Antwoord}\vfill\null\columnbreak\[#1\]\end{multicols}}

\begin{enumerate}[label=\emph{\arabic*})]
  \item
    \begin{enumerate}[label=\emph{\alph*})]
      \item
      \[5(x-3)=7x+8\]
      \[5x-15=7x+8\]
      \[-2x\cdot15=7x+8\]
      \[-2x=23\]
      \answer{x=-11\frac{1}{2}} 
        \item
        \[\frac{2}{3}x-2=\frac{1}{5}x-\frac{3}{5}\]
        \[\frac{2}{3}x=\frac{1}{5}x+1\frac{2}{5}\]
        \[\frac{2}{3}x-\frac{1}{5}x=1\frac{2}{5}\]
        \answer{\frac{7}{15}x=1\frac{2}{5}}

        \pagebreak

        \item
         \[2(3x-1)=x-(3x-14)\]
         \[6x-2=x-3x+14\]
         \[5x-2=-3x-14\]
         \[8x=-14\]
         \[8x=-12\]
         \answer{x=-1\frac{1}{2}}

         \item
           \[\frac{1}{6}a+4=\frac{1}{3}(a-3)-1\frac{1}{2}a\]
           \[\frac{1}{6}a+4=\frac{1}{3}a-1-1\frac{1}{2}a\]
           \[-\frac{2}{6}a+4=-1-1\frac{1}{2}a\]
           \[1\frac{1}{6}a+4=-1\]
           \[1\frac{1}{6}a=-5\]
           \answer{a=-4\frac{5}{6}}
    \end{enumerate}

  \item 
    \begin{figure}[ht]
      \centering
      \caption{De lijnen l,m,n}
      \label{fig:plot1}
    \begin{tikzpicture}
      \draw[thick,->] (0,0) -- (3.5,0)node[anchor=north west] {x};
      \draw[thick,->] (0,0) -- (0,7.5)node[anchor=south east] {y};
      \draw[step=1cm,black,very thin] (-4,-8) grid (4,8);
      \draw(0,-2) -- (4,4) node[anchor=west]{l};
      \draw(0,0) -- (4,-8) node[anchor=west]{m};
      \draw(0,2) -- (4,-1)node[anchor=west]{n};
    \end{tikzpicture}
  \end{figure}
    \[l: y=\frac{1}{2}x-2\]
    \[m: y=-2x\]
    \[n: y= -\frac{3}{4}\]
\end{enumerate}
\end{document}

% Created 2023-04-18 Tue 09:39
% Intended LaTeX compiler: pdflatex
\documentclass[11pt]{article}
\usepackage[utf8]{inputenc}
\usepackage[T1]{fontenc}
\usepackage{graphicx}
\usepackage{longtable}
\usepackage{wrapfig}
\usepackage{rotating}
\usepackage[normalem]{ulem}
\usepackage{amsmath}
\usepackage{amssymb}
\usepackage{capt-of}
\usepackage{hyperref}
\usepackage{pgfplots}
\usepgfplotslibrary{external}
\tikzexternalize
\date{\today}
\title{}
\hypersetup{
 pdfauthor={},
 pdftitle={},
 pdfkeywords={},
 pdfsubject={},
 pdfcreator={Emacs 28.2 (Org mode 9.5.5)}, 
 pdflang={English}}
\begin{document}

\tableofcontents


\section{Nooit meer fouten}
\label{sec:orgab22f78}
\subsection{1. Vraag verkeerd gelezen}
\label{sec:org414de86}
Oplossing: Na het maken van opdracht, vraag nog eens doorlezen.
\subsection{2. Bij exponentieel verband \(a(x-p)^n+q\) vergeten dat het \(\textbf{\underline{min}}\ p\) is}
\label{sec:org5353681}
Oplossing: aan het eind van \uline{elke} vraag controleren dat deze fout niet gemaakt is.
\subsection{3. Bij het over de \(=\) halen vergeten met \(-1\) te vermenigvuldigen}
\label{sec:orgfd1d22e}
\subsection{4. Bij vermenigvuldigingsfactoren maken een decimaal teveel of te weinig \(23\% \neq 1,023\), \(23\% = 1,23\)}
\label{sec:org886e8fa}
\subsection{5. Te weinig oplossingen bij vragen die er meerdere hebben}
\label{sec:org2441571}
Oplossing: als het om een functie gaat kun je het je voorstellen als een grafiek en dus zien of er meerdere oplossingen zijn

\begin{equation}
\begin{tikzpicture}
\begin{axis}[xmax=1.5,ymax=2,xmin=-1.5,ymin=0,]


\addplot[color=red, domain=-2:2, domain y=-2:2, samples=200]{x^4};

\addplot[color=green, domain=-2:2, domain y=-2:2, samples=200]{x^10};

\node[label={180:{(-1,1)}},circle,fill,inner, set=1pt] at (axis cs:-1,1);
\node[label={180:{(1,1)}},circle,fill,inner, set=1pt] at (axis cs:1,1);


\end{axis}
\end{tikzpicture}
\end{equation}
\end{document}
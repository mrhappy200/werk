% Created 2023-11-12 Sun 17:05
% Intended LaTeX compiler: pdflatex
\documentclass[11pt]{article}
\usepackage[utf8]{inputenc}
\usepackage[T1]{fontenc}
\usepackage{graphicx}
\usepackage{longtable}
\usepackage{wrapfig}
\usepackage{rotating}
\usepackage[normalem]{ulem}
\usepackage{amsmath}
\usepackage{amssymb}
\usepackage{capt-of}
\usepackage{hyperref}
\usepackage{chemfig}
\date{\today}
\title{}
\hypersetup{
 pdfauthor={},
 pdftitle={},
 pdfkeywords={},
 pdfsubject={},
 pdfcreator={Emacs 29.1 (Org mode 9.6.6)}, 
 pdflang={English}}
\begin{document}

\setcounter{tocdepth}{3}
\tableofcontents

\begin{verbatim}
date # last updated
\end{verbatim}

\begin{verbatim}
Sun 12 Nov 2023 17:05:54 CET
\end{verbatim}

\section{Data}
\label{sec:org81b92f1}
\begin{center}
\begin{tabular}{lll}
datum & vak & stof\\[0pt]
ma 27 nov & nat & H1/H2\\[0pt]
ma 27 nov & schk & H1 tm H3.4\\[0pt]
di 28 nov & fatl & Unite 1 / Unite 2\\[0pt]
wo 29 nov & netl & classroom\\[0pt]
wo 29 nov & wisB & H2/H4\\[0pt]
vr 31 nov & entl & essay\\[0pt]
vr 31 nov & maat & ?????????\\[0pt]
\end{tabular}
\end{center}

\section{Vakken}
\label{sec:orgb4649f0}
\subsection{{\bfseries\sffamily DONE} Engels}
\label{sec:orga552bf0}
\subsubsection{Stof}
\label{sec:org5fa5dca}
Essay van 250-300 woorden over een actueel onderwerp. \textbf{Spellingcontrole MAG!}
\subsubsection{Plan}
\label{sec:orgf9f89c7}
\begin{enumerate}
\item {\bfseries\sffamily DONE} Oefenen in de les
\label{sec:org2f65b30}
\end{enumerate}
\subsection{{\bfseries\sffamily TODO} Natuurkunde}
\label{sec:org9701f1a}
\subsubsection{Stof}
\label{sec:org033ba52}
H1/H2
\subsubsection{Plan}
\label{sec:orgacfa0e4}
\begin{enumerate}
\item {\bfseries\sffamily TODO} H1 leren
\label{sec:orgd95fe45}
\begin{itemize}
\item[{$\square$}] Paragraaf 1.2 Energie en vermogen
\item[{$\square$}] Paragraaf 1.3 Spanning en stroomsterkte
\item[{$\square$}] Paragraaf 1.4 Weerstand
\end{itemize}
\item {\bfseries\sffamily TODO} H2 leren
\label{sec:orgc73ee69}
\begin{itemize}
\item[{$\square$}] Paragraaf 2.2 Kracht verandert snelheid
\item[{$\square$}] Paragraaf 2.3 Versnellen en vertragen
\item[{$\square$}] Paragraaf 2.4 Afstand en beweging
\item[{$\square$}] Paragraaf 2.5 Vallen
\end{itemize}
\end{enumerate}
\subsection{{\bfseries\sffamily TODO} Nederlands}
\label{sec:orgcdc0cd3}
\subsubsection{Stof}
\label{sec:orgb649502}
Tekstbegrip Classroom
\subsubsection{Plan}
\label{sec:org6925697}
\begin{enumerate}
\item {\bfseries\sffamily TODO} In de les de theorie krijgen
\label{sec:org65ca094}
\item {\bfseries\sffamily TODO} \ldots{}
\label{sec:orgc5112fd}
\end{enumerate}
\subsection{{\bfseries\sffamily DOING} Scheikunde}
\label{sec:orgf175b14}
\subsubsection{Stof}
\label{sec:orgc06c0dd}
H1 t/m H3.4
\subsubsection{Plan}
\label{sec:orgc9e3133}
\begin{enumerate}
\item {\bfseries\sffamily DOING} H1 Scheiden en reageren
\label{sec:org4893c8f}
\begin{itemize}
\item[{$\boxminus$}] Paragraaf 1.1 Zuivere stof en mengsel
\begin{itemize}
\item[{$\boxminus$}] Zuivere stof of mengsel
\item[{$\boxminus$}] Oplossing of emulsie
\end{itemize}
\item[{$\square$}] Paragraaf 1.2 Scheidingsmethoden
\begin{itemize}
\item[{$\square$}] Destillatie
\item[{$\square$}] Zand
\item[{$\square$}] Adsorptie
\item[{$\square$}] Chromatografie
\end{itemize}
\item[{$\boxminus$}] Paragraaf 1.3 Chemische reacties
\begin{itemize}
\item[{$\boxminus$}] Exotherm of Endotherm
\end{itemize}
\item[{$\square$}] Paragraaf 1.4 Snelheid van een reactie
\begin{itemize}
\item[{$\square$}] Reactiesnelheid
\item[{$\square$}] Concentratie en reactiesnelheid
\item[{$\square$}] Temeratuur en reactiesnelheid
\item[{$\boxminus$}] Katalysatoren
\end{itemize}
\end{itemize}
\item {\bfseries\sffamily TODO} H2 Bouwstenen van stoffen
\label{sec:org6039b03}
\begin{itemize}
\item[{$\square$}] Paragraaf 2.1 Periodiek systeem
\begin{itemize}
\item[{$\square$}] Ijzer wordt roest
\item[{$\square$}] De grootte van een molecuul
\end{itemize}
\item[{$\square$}] Paragraaf 2.2 Ionen
\item[{$\square$}] Paragraaf 2.3 Massa's van bouwstenen
\item[{$\square$}] Paragraaf 2.4 De mol
\end{itemize}
\item {\bfseries\sffamily TODO} H3 Moleculaire stoffen
\label{sec:org41d9667}
\begin{itemize}
\item[{$\square$}] Paragraaf 3.1 De bouw van stoffen
\begin{itemize}
\item[{$\square$}] Elektrische geleiding
\end{itemize}
\item[{$\square$}] Paragraaf 3.2 Binding in moleculen
\item[{$\square$}] Paragraaf 3.3 Binding tussen moleculen
\begin{itemize}
\item[{$\square$}] Verwarmen van moleculaire stoffen
\item[{$\square$}] Bindingen in water
\item[{$\square$}] Polariteit en lading
\end{itemize}
\item[{$\square$}] Paragraaf 3.4 Moleculaire stoffen mengen
\begin{itemize}
\item[{$\square$}] Jood in water, jood in wasbenzine
\item[{$\square$}] Oplossen of niet
\item[{$\square$}] De ammoniakfontein
\item[{$\square$}] Mentos in cola
\end{itemize}
\end{itemize}
\end{enumerate}
\subsection{{\bfseries\sffamily TODO} WisB}
\label{sec:org7a70e9e}
\subsubsection{Stof}
\label{sec:orgc8caaac}
H2/H4
\subsubsection{Plan}
\label{sec:orgc65c68e}
\begin{enumerate}
\item {\bfseries\sffamily TODO} H2
\label{sec:org5e74ac8}
\item {\bfseries\sffamily TODO} H4
\label{sec:org8e6625e}
\end{enumerate}
\subsubsection{Notities}
\label{sec:orgeef6071}
\begin{itemize}
\item[{$\square$}] Modulusvergelijkingen
\item[{$\square$}] 
\end{itemize}
\subsection{{\bfseries\sffamily TODO} Frans}
\label{sec:org1066c6e}
\subsubsection{Stof}
\label{sec:org1c05206}
\begin{itemize}
\item Vocabulaire Unite 1 : Lire + Vocabulaire thématique ( NF)
\item Vocabulaire Unité 2
\item Grammaire I: l' adverbe
\item Grammaire II: Verbes -RE  (descendre, vendre, attendre, répondre) + pouvoir + vouloir
\item Grammaire III: les possessifs  (zelfstandig bezittelijk voornaamwoord niet in toets)
\end{itemize}
\subsubsection{Plan}
\label{sec:org7603f1e}
\begin{itemize}
\item[{$\square$}] Vocabulaire Unite 1
\item[{$\square$}] Vocabulaire Unite 2
\item[{$\square$}] Grammaire I
\begin{itemize}
\item[{$\square$}] l'adverbe
\end{itemize}
\item[{$\square$}] Grammaire II
\begin{itemize}
\item[{$\square$}] Verbes -RE
\item[{$\square$}] Pouvoir
\item[{$\square$}] Vouloir
\end{itemize}
\item[{$\square$}] Grammaire III
\begin{itemize}
\item[{$\square$}] les possesifs (niet bezittelijk voornaamwoord)
\end{itemize}
\end{itemize}
\end{document}